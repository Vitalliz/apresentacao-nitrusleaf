\documentclass{beamer}
\usepackage[utf8]{inputenc}
\usepackage{lmodern}
\usepackage[brazil]{babel}
\usetheme{Madrid}
\usepackage{multimedia}

% -------------------------------------------------
% Identidade Visual - Centro Paula Souza
% -------------------------------------------------
\definecolor{cpsred}{RGB}{153,0,0}      % Vermelho institucional
\definecolor{cpsgray}{RGB}{85,85,85}    % Cinza técnico

\usecolortheme[named=cpsred]{structure}
\setbeamercolor{title}{fg=white,bg=cpsred}
\setbeamercolor{frametitle}{fg=white,bg=cpsred}
\setbeamercolor{structure}{fg=cpsred}
\setbeamercolor{normal text}{fg=black,bg=white}
\setbeamercolor{itemize item}{fg=cpsred}

\usepackage{ragged2e}
\usepackage{hyperref}

\setbeamertemplate{headline}{} 

% -------------------------------------------------
% Rodapé padrão
% -------------------------------------------------
\setbeamertemplate{footline}[frame number]
\addtobeamertemplate{footline}{
  \hfill\usebeamercolor[fg]{frametitle}{\hspace{-1.5cm}\scriptsize }\hspace{1cm}
}{}

% -------------------------------------------------
% Informações principais
% -------------------------------------------------
\title[Classificação de Manganês e Cobre na Folha da Mexerica, Orientado por Redes Neurais]{\textbf{Classificação de Manganês e Cobre na Folha da Mexerica, Orientado por Redes Neurais}}
\author{Amanda Vitória Alves Freitas \\ Lucas Gomes Fagundes \\ Valéria de Freitas}
\date{}

% -------------------------------------------------
% Documento
% -------------------------------------------------
\begin{document}

% Capa
\begin{frame}
    \centering
    \vspace{1cm}
    {\color{cpsred}\Huge\textbf{Classificação de Manganês e Cobre na Folha da Mexerica, Orientado por Redes Neurais}}\\[0.8cm]
    {\Large Amanda Vitória Alves Freitas \\ Lucas Gomes Fagundes \\ [0.cm]Valéria de Freitas}\\[0.4cm]
    \textcolor{cpsgray}{Centro Paula Souza}\\[0.2cm]
    \textcolor{cpsgray}{FATEC Registro}
\end{frame}

% Sumário
\begin{frame}{Sumário}
\tableofcontents
\end{frame}

% -------------------------------------------------
% SEÇÕES E SLIDES
% -------------------------------------------------

\section{Pitch}
\begin{frame}{Pitch}
\justifying
Apresentação breve e objetiva do tema, destacando sua relevância, objetivos principais e metodologia adotada.

\end{frame}

\section{Problematização}
\begin{frame}{Problematização}
\justifying
Falta de padronização (itálico de periódicos, DOI, formatação).  
Incoerência entre o que é citado no texto e o que aparece na lista.  
URLs soltas, entradas incompletas e referências fora da norma.
\end{frame}

\section{Estado da Arte}
\begin{frame}{Estado da Arte}
\justifying
Citações provenientes de páginas não científicas, ausência de fontes oficiais, artigos revisados por pares ou documentos institucionais confiáveis.
\end{frame}

\section{Objetivo}
\begin{frame}{Objetivo}
\justifying
Ausência de dados quantitativos e de referências.  
Organização confusa e, em muitos casos, a tecnologia é apresentada antes da definição clara do problema.
\end{frame}

\section{Metodologia}
\begin{frame}{Metodologia}
\justifying
Objetivos gerais pouco operacionais, específicos ausentes ou mal apresentados.  
Há redundâncias e inversão entre o que é produto e o que é método.
\end{frame}

\section{Aplicação Prática}
\begin{frame}{Aplicação Prática}
\justifying
Mistura de etapas, ferramentas e resultados.  
Falta de clareza sobre o fluxo de desenvolvimento, separação entre metodologia de pesquisa e de implementação, e presença de trechos do modelo sem conteúdo autoral.
\end{frame}

\section{Resultados}
\begin{frame}{Resultados}
\justifying
Figuras muito pequenas ou sem legenda.  
Ausência de fluxogramas obrigatórios.  
Tabelas e seções de exemplo do modelo não substituídas adequadamente.
\end{frame}

\section{Conclusão}
\begin{frame}{Conclusão}
\justifying
Pedidos de “citação ao final do parágrafo” ignorados.  
Falta de identificação clara da fonte de dados e lacunas entre citações e lista de referências.
\end{frame}

\end{document}
