\documentclass{beamer}
\usepackage[utf8]{inputenc}
\usepackage{lmodern}
\usepackage[brazil]{babel}
\usetheme{Madrid}
\usepackage{multimedia}
\usepackage{ragged2e}
\usepackage{hyperref}
\usepackage{xurl}                 % permite quebra "inteligente" em qualquer ponto
\hypersetup{breaklinks=true}
\usepackage{setspace}
\Urlmuskip=0mu plus 2mu
\def\UrlBreaks{\do\/\do\-\do\_\do\&\do\?\do\=\do\%}


% -------------------------------------------------
% Identidade Visual - Centro Paula Souza
% -------------------------------------------------
\definecolor{cpsred}{RGB}{153,0,0}      % Vermelho institucional
\definecolor{cpsgray}{RGB}{85,85,85}    % Cinza técnico

\usecolortheme[named=cpsred]{structure}
\setbeamercolor{title}{fg=white,bg=cpsred}
\setbeamercolor{frametitle}{fg=white,bg=cpsred}
\setbeamercolor{structure}{fg=cpsred}
\setbeamercolor{normal text}{fg=black,bg=white}
\setbeamercolor{itemize item}{fg=cpsred}
% --- Aplicar o tema CPS aos blocos ---
\setbeamercolor{block title}{fg=white,bg=cpsred}
\setbeamercolor{block body}{fg=black,bg=white!98!cpsred}
\setbeamercolor{exampleblock title}{fg=white,bg=cpsred}
\setbeamercolor{exampleblock body}{fg=black,bg=white!98!cpsred}
\setbeamercolor{alertblock title}{fg=white,bg=cpsred}
\setbeamercolor{alertblock body}{fg=black,bg=white!98!cpsred}

% --- Forçar cores personalizadas nos exampleblocks ---
\setbeamercolor*{block title example}{use=structure,fg=white,bg=cpsred}
\setbeamercolor*{block body example}{fg=black,bg=white!98!cpsred}

\setbeamercolor*{block title alerted}{use=structure,fg=white,bg=cpsred}
\setbeamercolor*{block body alerted}{fg=black,bg=white!98!cpsred}

\setbeamercolor*{block title}{use=structure,fg=white,bg=cpsred}
\setbeamercolor*{block body}{fg=black,bg=white!98!cpsred}

% --- Bordas arredondadas e sombra leve ---
\setbeamertemplate{blocks}[rounded][shadow=true]

\usepackage{ragged2e}
\usepackage{hyperref}

\setbeamertemplate{headline}{} 

\usepackage{xurl}
\hypersetup{breaklinks=true}
\def\UrlBreaks{\do\/\do\-\do\_\do\&\do\?\do\=\do\%} % prefere quebrar em / - _ & ? = %
\Urlmuskip=0mu plus 2mu


% -------------------------------------------------
% Rodapé padrão
% -------------------------------------------------
\setbeamertemplate{footline}{
  \leavevmode%
  \hbox{%
    \begin{beamercolorbox}[wd=\paperwidth,ht=0.8cm,dp=0.2cm,leftskip=0.5cm,rightskip=0.5cm]{footline}
      \color{white}
      \raisebox{-0.5em}{
        \includegraphics[height=0.6cm]{img/Fatec-Logo.png}\hspace{0.3cm}
        \includegraphics[height=0.8cm]{img/Cps-Logo.png}\hspace{0.3cm}
      }
      \textbf{\insertshortauthor}
      \hfill
      \raisebox{1em}{\insertframenumber{} / \inserttotalframenumber}
    \end{beamercolorbox}%
  }%
}
\setbeamercolor{footline}{bg=cpsred,fg=white}



% -------------------------------------------------
% Documento
% -------------------------------------------------
\begin{document}

% Capa
\begin{frame}[plain]
    \centering

    % --- Espaçamento inicial para centralizar a logo ---
    \vspace*{2.5cm}

    % --- Logo centralizada ---
    \includegraphics[width=10cm]{img/Nitrusleaf-Logo.png}

    % --- Espaçamento até os textos de rodapé ---
    \vspace*{2.8cm}

    % --- Rodapé dividido: autores à esquerda e instituição à direita ---
    \begin{minipage}[t]{0.45\textwidth}
        \raggedright
        \textcolor{cpsgray}{Amanda Vitória Alves Freitas}\\
        \textcolor{cpsgray}{Lucas Gomes Fagundes}\\
        \textcolor{cpsgray}{Valéria de Freitas}
    \end{minipage}
    \hfill
    \begin{minipage}[t]{0.45\textwidth}
        \raggedleft
        \textcolor{cpsgray}{\textbf{Centro Paula Souza}}\\
        \textcolor{cpsgray}{FATEC Registro}
    \end{minipage}

\end{frame}


% Sumário
\begin{frame}{Agenda}
\tableofcontents
\end{frame}

% -------------------------------------------------
% SEÇÕES E SLIDES


% Pitch-------------------------------------------------
\section{Pitch}
\begin{frame}{Pitch}
\justifying
Apresentação breve e objetiva do tema, destacando sua relevância, objetivos principais e metodologia adotada.

\end{frame}

% Problematização-------------------------------------------------
\section{Problematização}
\begin{frame}{Problematização}
\justifying

\centering{\textbf{Percentual das laranjeiras com greening por setor e região}}

\vspace{1mm}
\begin{center}
\includegraphics[width=0.58\linewidth]{img/Greening-incidencia-prob.png}

\vspace{1mm}
{\footnotesize Figura 1 – Divisão do cinturão citrícola em 5 setores e 12 regiões. \\Fonte: Fundecitrus, São Paulo 2025\textsuperscript{1}}
\end{center}

%------- referência -------
\vspace{-1mm}
\noindent\begin{minipage}[t]{0.96\linewidth}
\RaggedRight
% entrelinha bem compacta:
\begin{spacing}{0.35}
{\fontsize{4.5}{6.8}\selectfont  % 4pt com entrelinha 6.8pt (bem junto)
\setlength{\parskip}{0pt}\setlength{\parsep}{0pt}\setlength{\partopsep}{0pt}%
\sloppy % permite quebras mais agressivas sem “overfull”

\textsuperscript{1}\,Fundecitrus. \textit{Levantamento de greening no Cinturão Citrícola de São Paulo e Triângulo/Sudoeste Mineiro}. 2025.
Disponível em: \url{https://www.fundecitrus.com.br/wp-content/uploads/2025/09/Levantamento-de-doencas-2025_Resumo-greening.pdf}.
Acesso em: 28 out. 2024.
}
\end{spacing}
\end{minipage}

\end{frame}

% Estado da Arte 1-------------------------------------------------
\section{Estado da Arte}  
\begin{frame}{Estado da Arte}
\justifying

\centering{\textbf{Abordagens Baseadas em Visão Computacional, Aprendizado de Máquina e 
Aprendizado Profundo para Identificação de Deficiências Nutricionais em Culturas
}}

\vspace{1mm}
\begin{center}
\includegraphics[width=0.58\linewidth]{img/Estado-da-arte-1.png}

\vspace{1mm}
{\footnotesize Figura 2 – Divisão do cinturão citrícola em 5 setores e 12 regiões. \\Fonte: Muthusamy e Ramu, 2023\textsuperscript{2}}
\end{center}

%------- referência -------
\vspace{-1mm}
\noindent\begin{minipage}[t]{0.96\linewidth}
\RaggedRight
% entrelinha bem compacta:
\begin{spacing}{0.35}
{\fontsize{4.5}{6.8}\selectfont  % 4pt com entrelinha 6.8pt (bem junto)
\setlength{\parskip}{0pt}\setlength{\parsep}{0pt}\setlength{\partopsep}{0pt}%
\sloppy % permite quebras mais agressivas sem “overfull”

\textsuperscript{2}\,MUTHUSAMY, Sudhakar; RAMU, Swarna Priya. \textit{Computer Vision Based Machine Learning and Deep Learning Approaches for Identification of Nutrient Deficiency in Crops: A Survey.}. 2023.
Disponível em: \url{https://doi.org/10.46488/NEPT.2023.v22i03.025}. Acesso em: 28 out. 2024.
}
\end{spacing}
\end{minipage}

\end{frame}

% Estado da Arte 2-----------------------------------------------
\begin{frame}{Estado da Arte}
\justifying

\centering{\textbf{Detecção de Doenças e Deficiências Nutricionais em Plantas Baseada em Processamento de Imagens}}

\vspace{1mm}
\begin{center}
\includegraphics[width=0.58\linewidth]{img/Estado-da-arte-2.png}

\vspace{1mm}
{\footnotesize Figura 3 – Esquema do pipeline para detecção de doenças em plantas baseada em imagens. Fonte: Ghorai et al., 2021\textsuperscript{3}}
\end{center}

%------- referência -------
\vspace{-1mm}
\noindent\begin{minipage}[t]{0.96\linewidth}
\RaggedRight
% entrelinha bem compacta:
\begin{spacing}{0.35}
\fontsize{4.5}{6.8}\selectfont  % 4pt com entrelinha 6.8pt (bem junto)
\setlength{\parskip}{0pt}\setlength{\parsep}{0pt}\setlength{\partopsep}{0pt}%
\sloppy % permite quebras mais agressivas sem “overfull”

\textsuperscript{3}\,GHORAI, Anirban; et al. \textit{Image Processing Based Detection of Diseases and Nutrient Deficiencies in Plants.}. 2021.
Disponível em: \url{https://www.researchgate.net/publication/349707825_Image_Processing_Based_Detection_of_Diseases_and_Nutrient_Deficiencies_in_Plants}. Acesso em: 28 out. 2024.

\end{spacing}
\end{minipage}

\end{frame}

% Estado da Arte 3-----------------------------------------------
\begin{frame}{Estado da Arte}
\justifying
\centering{\textbf{Um Estudo Comparativo de Deep CNN Na Previsão e Classificação de Deficiências de Macronutrientes no Desenvolvimento de Plantas de Tomate}}

\begin{columns}[T,onlytextwidth]
  % Coluna da imagem
  \column{0.52\textwidth}
  \centering
  \includegraphics[width=\linewidth]{img/Estado-da-arte-3.jpg}
  \vspace{0.5mm}

  % Coluna do texto
  \column{0.45\textwidth}
  \raggedright

  {\scriptsize
  \vspace{1.5em}
  \setlength{\parskip}{0pt}\setlength{\baselineskip}{9.5pt}%
  Figura 4 – Resultados de Previsão do Inception-ResNet v2.\\
  (a–c) Previsão de Deficiência de Cálcio;\\
  (d–f) Previsão de Carência de Potássio;\\
  (g,h) Previsão de Falta de Nitrogênio.\par
  \vspace{1mm}
  \emph{Fonte:} Tran et al., 2019\textsuperscript{3}
  }
\end{columns}


                %------- referência -------
\vspace{-1mm}
\noindent\begin{minipage}[t]{0.96\linewidth}
\RaggedRight
% entrelinha bem compacta:
\begin{spacing}{0.35}
{\fontsize{4.5}{6.8}\selectfont  % 4pt com entrelinha 6.8pt (bem junto)
\setlength{\parskip}{0pt}\setlength{\parsep}{0pt}\setlength{\partopsep}{0pt}%
\sloppy % permite quebras mais agressivas sem “overfull”

\textsuperscript{3}\,Trung-Tin et al.\textit{A Comparative Study of Deep CNN in Forecasting and Classifying the Macronutrient Deficiencies on Development of Tomato Plant}. 2019.
Disponível em: \url{https://doi.org/10.46488/NEPT.2023.v22i03.025}. Acesso em: 28 out. 2024.
}
\end{spacing}
\end{minipage}

\end{frame}


% Objetivo-------------------------------------------------
\section{Objetivo}
\begin{frame}{Objetivo}
\justifying
Diagnosticar deficiência de Cobre ou Manganês na folha 
da Mexerica \textit{(Citrus Reticulata)}, através de Inteligência Artificial

\vspace{0.5cm} % <-- pequeno gap entre o texto e o bloco

\begin{block}{Objetivos Específicos}
  \begin{itemize}\setlength{\itemsep}{6pt} % <-- aumenta o espaço entre os itens
    \item Construir um banco de imagens de folhas com deficiências;
    \item Treinar e validar uma \textbf{CNN} (Redes Neurais Convolucionais);
    \item Implementar um protótipo funcional com captura via smartphone;
    \item Registrar dados em um histórico de acompanhamento;
    \item Criar módulo de recomendações técnicas com base nos resultados.
  \end{itemize}
\end{block}

\end{frame}


% ------------------------------------------------- % 2º slide
\begin{frame}{Objetivo}
\justifying
% --- Texto introdutório ---
\centering{Motivações e Benefícios:}
\vspace{0.4cm}

% --- Blocos organizados em colunas ---
\begin{columns}[T] % alinhamento pelo topo
  % --- Coluna 1 ---
  \column{0.45\textwidth}
  \begin{exampleblock}{\centering Facilidade de Diagnóstico}
    IA acessível para identificar deficiências via smartphone.
  \end{exampleblock}

  \vspace{0.3cm}

  \begin{exampleblock}{\centering Eficiência Comparada a Outros Métodos}
    Diagnóstico rápido e preciso sem necessidade de laboratório.
  \end{exampleblock}

  % --- Coluna 2 ---
  \column{0.45\textwidth}
  \begin{exampleblock}{\centering Registro e Visualização}
    Mapas e gráficos interativos para monitoramento das plantas.
  \end{exampleblock}

  \vspace{0.45cm}

  \begin{exampleblock}{\centering Sustentabilidade Agrícola}
    Uso racional de insumos e redução de perdas na produção.
  \end{exampleblock}
\end{columns}

\end{frame}

% -------------------------------------------------
\section{Metodologia}
\begin{frame}{Metodologia}
\justifying
Objetivos gerais pouco operacionais, específicos ausentes ou mal apresentados.  
Há redundâncias e inversão entre o que é produto e o que é método.
\end{frame}

% -------------------------------------------------
\section{Aplicação Prática}
\begin{frame}{Aplicação Prática}
\justifying
% --
\centering Apresentação prática do projeto:
\vspace{0.2cm}

% --- Blocos organizados em colunas ---
\begin{columns}[T] % alinhamento pelo topo
  % --- Coluna 1 ---
  \column{0.4\textwidth}
  \begin{exampleblock}{\centering Protótipo desenvolvido no \textbf{Figma}}
    \centering
    \includegraphics[width=2cm]{img/Figma-Logo.png}
    \vspace{0.2cm}
  \end{exampleblock}

  % --- Coluna 2 ---
  \column{0.45\textwidth}
  \begin{exampleblock}{\centering Aplicativo móvel implementado em \textbf{React Native}}
    \centering
    \includegraphics[width=2cm]{img/React-Logo.png}
    \vspace{0.3cm}
  \end{exampleblock}

\end{columns}

\end{frame}

% -------------------------------------------------
\section{Resultados}
\begin{frame}{Resultados}
\justifying
Figuras muito pequenas ou sem legenda.  
Ausência de fluxogramas obrigatórios.  
Tabelas e seções de exemplo do modelo não substituídas adequadamente.
\end{frame}

% -------------------------------------------------
\section{Conclusão}
\begin{frame}{Conclusão}
\justifying
Pedidos de “citação ao final do parágrafo” ignorados.  
Falta de identificação clara da fonte de dados e lacunas entre citações e lista de referências.
\end{frame}

\end{document}
